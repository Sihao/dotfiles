%%%%%%%%%%%%%%%%%%%%%%%%%%%%%%%%%%%%%%%%%
% Simple Sectioned Essay Template
% LaTeX Template
%
% This template has been downloaded from:
% http://www.latextemplates.com
%
%
%%%%%%%%%%%%%%%%%%%%%%%%%%%%%%%%%%%%%%%%%

%----------------------------------------------------------------------------------------
%	PACKAGES AND OTHER DOCUMENT CONFIGURATIONS
%----------------------------------------------------------------------------------------
%!TEX root = 00Main.tex

%----------------------------------------------------------------------------------------
%	PACKAGES AND OTHER DOCUMENT CONFIGURATIONS
%----------------------------------------------------------------------------------------
\PassOptionsToPackage{table}{xcolor}

\documentclass[12pt]{article} % Default font size is 12pt, it can be changed here

\usepackage[margin=1.2in,footskip=1cm]{geometry} % Required to change the page size to A4
\geometry{a4paper} % Set the page size to be A4 as opposed to the default US Letter
\usepackage{textcomp}
\usepackage{lipsum}
\usepackage{graphicx} % Required for including pictures
\usepackage{amsmath}
\DeclareMathOperator*{\argmin}{arg\,minimize}
\usepackage{courier}
\usepackage{bm} %boldsymbol mathmode
% \usepackage{nccmath} %Smaller fonts for big matrices
\usepackage{float} % Allows putting an [H] in \begin{figure} to specify the exact location of the figure
\usepackage{wrapfig} % Allows in-line images such as the example fish picture
% \usepackage{pgfgantt}
\usepackage{subcaption}
\usepackage{parskip} %Read in blank lines in text (allows for paragraphs)
\usepackage[super,numbers,sort&compress]{natbib}
\renewcommand*{\thefootnote}{(\arabic{footnote})} %Change footnote styling to avoid confusion with citations
\usepackage{array}
\usepackage{tikz}
\usepackage{xcolor}
\usepackage{listings} %Typesetting code
\usepackage{fancyhdr}	%Headers and footers
\usepackage{titlesec}	%Alter titles
\usepackage{sectsty}	
\usepackage{calc}
\usepackage{abstract}
\usepackage{appendix}
\usepackage[usenames,dvipsnames]{pstricks}
\usepackage{epsfig}
\usepackage{multirow}
\usepackage{longtable}
\usepackage[nottoc,numbib]{tocbibind}
\usepackage{multicol}
\usepackage{nicefrac}
\usepackage[greek.ancient,english]{babel}
\usepackage{gensymb}
\usepackage{mcode}
\usepackage{scrextend}
\usepackage{dsfont}
\usepackage{tikz}
\usepackage{booktabs}
\usepackage{hyperref}   % Hyperlinking
\usepackage{cleveref} % Has to be loaded last

%%%%%%%%%%%%%%%%%%%%%%%%%%%%  Defining colours  %%%%%%%%%%%%%%%%%%%%%%%%%%%%

\definecolor{BioEngTeal}{rgb}{0.43529411764,0.79607843137,0.82352941176}
\definecolor{BioEngGrey}{rgb}{0.56862745098,0.5725490196,0.58431372549}
\definecolor{mygreen}{RGB}{28,172,0} % color values Red, Green, Blue
\definecolor{mylilas}{RGB}{170,55,241}
\definecolor{RubineRed}{RGB}{206,0,88}

%%%%%%%%%%%%%%%%%%%%%%%%%%%%%%%%%%%%%%%%%%%%%%%%%%%%%%%%%%%%%%%%%%%%%%%%%%%%%

%%%%%%%%%%%%%%%%%%%%%%%%%%%  New environment for typesetting big matrices  %%%%%%%%%%%%%%%%%%%%%%%%

\newenvironment{mbmatrix}{\begin{medsize}\begin{bmatrix}}%
{\end{bmatrix}\end{medsize}}%

%%%%%%%%%%%%%%%%%%%%%%%%%%%%%%%%%%%%%%%%%%%%%%%%%%%%%%%%%%%%%%%%%%%%%%%%%%%%%%%%%%%%%%%%%%%%%%%%%%%%%

%%%%%%%%%%%%%%%%  Change colours of sections, subsections, figures and tables  %%%%%%%%%%%%%%%%%

\sectionfont{\color{BioEngTeal}}  % sets colour of sections
\subsectionfont{\color{BioEngGrey}}  % sets colour of subsections
\renewcommand{\abstractnamefont}{\color{BioEngTeal}\textbf}


\renewcommand{\bibname}{References}
\renewcommand{\thefigure}{\textcolor{BioEngTeal}{\bfseries\itshape\thesection.\arabic{figure}\,}}
\renewcommand{\thesubfigure}{\textcolor{BioEngTeal}{\bfseries\alph{subfigure}}}
\renewcommand{\figurename}{\textcolor{BioEngTeal}{\bfseries\itshape Figure}}
\renewcommand{\thetable}{\textcolor{BioEngTeal}{\bfseries\itshape\thesection.\arabic{table}}}
\renewcommand{\tablename}{\textcolor{BioEngTeal}{\bfseries\itshape Table}}
\renewcommand{\theequation}{\textcolor{BioEngTeal}{\bfseries\itshape\thesection.\arabic{equation}\,}}

\addto\captionsenglish{\renewcommand{\figurename}{\textcolor{BioEngTeal}{\bfseries\itshape Figure}}}
\addto\captionsenglish{\renewcommand{\tablename}{\textcolor{BioEngTeal}{\bfseries\itshape Table}}}
%%%%%%%%%%%%%%%%%%%%%%%%%%%%%%%%%%%%%%%%%%%%%%%%%%%%%%%%%%%%%%%%%%%%%%%%%%%%%%%%%%%%%%%%%%%%%%%%%




%%%%%%%%%%%%%%%%%%%%%%%%%%%%%%% Define macro for C++ symbol  %%%%%%%%%%%%%%%%%%%%%%%%%%%%%%%%%%%%

\newcommand{\CC}{C\nolinebreak\hspace{-.05em}\raisebox{.4ex}{\bf +}\nolinebreak\hspace{-.10em}\raisebox{.4ex}{\bf +}\,\,}
\def\CC{{C\nolinebreak[4]\hspace{-.05em}\raisebox{.4ex}{\bf ++}}\,\,}

%%%%%%%%%%%%%%%%%%%%%%%%%%%%%%%%%%%%%%%%%%%%%%%%%%%%%%%%%%%%%%%%%%%%%%%%%%%%%%%%%%%%%%%%%%%%%%%%%


%%%%%%%%%%%%%%%%%%%%%%%%%%  Defining norm symbol  %%%%%%%%%%%%%%%%%%%%%%

\newcommand{\norm}[1]{\left\lVert#1\right\rVert}

%%%%%%%%%%%%%%%%%%%%%%%%%%%%%%%%%%%%%%%%%%%%%%%%%%%%%%%%%%%%%%%%%%%%%%%%


%%%%%%%%%%%%%%%%%%%%%%%%%%%%%%%%%  Defining footer  %%%%%%%%%%%%%%%%%%%%%%%%%%%%%%%%%%

\pagestyle{fancy}
\fancyhf{}
\renewcommand{\footrulewidth}{0.5pt}
\renewcommand{\headrulewidth}{0pt}

\fancyfoot[R]{\color{BioEngTeal}\textbf{\thepage}}
\fancyfoot[L]{\raisebox{-.75\height}{\includegraphics[scale=1]{Pictures/bioengineering_logo_left.eps}}}

%%%%%%%%%%%%%%%%%%%%%%%%%%%%%%%%%%%%%%%%%%%%%%%%%%%%%%%%%%%%%%%%%%%%%%%%%%%%%%%%%%%%%%





%%%%%%%%%%%%%%%%%%%%%%%  Change font to Helvetica %%%%%%%%%%%%%%%%%%%%%%%%%%%

% % Include the helvet package
% \usepackage{helvet}

% % Set the default font to be sans-serif
% \renewcommand*{\familydefault}{\sfdefault}
% \makeatletter
% \newcommand{\thickhline}{%
%     \noalign {\ifnum 0=`}\fi \hrule height 2pt
%     \futurelet \reserved@a \@xhline
% }
% \newcolumntype{"}{@{\hskip\tabcolsep\vrule width 1pt\hskip\tabcolsep}}
% \makeatother
%%%%%%%%%%%%%%%%%%%%%%%%%%%%%%%%%%%%%%%%%%%%%%%%%%%%%%%%%%%%%%%%%%%%%%%%%%%%%%%%%%%%%

%%%%%%%%%%%%%%%%%%%%%%%  Fonts  %%%%%%%%%%%%%%%%%%%%%%%%%%%
\usepackage[scale=.75]{sourcecodepro} % monospace font
\usepackage{courier}
\renewcommand{\ttfamily}{pcr}
\usepackage{gfsbaskerville}
\usepackage[osf]{Baskervaldx} % tosf in text, tlf in math
\usepackage[T1]{fontenc}
% \usepackage[baskervaldx,cmintegrals,bigdelims,vvarbb]{newtxmath} % math italic letters from Baskervaldx
% \usepackage[cal=boondoxo]{mathalfa} % mathcal from STIX, unslanted a bit
% \usepackage{ebgaramond-maths}
%%%%%%%%%%%%%%%%%%%%%%%%%%%%%%%%%%%%%%%%%%%%%%%%%%%%%%%%%%%%%%%%%%%%%%%%%%%%%%%%%%%%%

\linespread{1.2} % Line spacinggit gitg

\setlength\parindent{0pt} % Uncomment to remove all indentation from paragraphs

\graphicspath{{Pictures/}} % Specifies the directory where pictures are stored

%%%%%%%%%%%%%%%%%%%%%%%%%%%%%%%%%%%%%%%%%%%%%%%%%%%%%%%%%%%%%%%%%%%%%%%%%%%%%%%%%%%%%%



%%%%%%%%%%%%%%%%%%%%%%%%%%%%%  Typesetting code  %%%%%%%%%%%%%%%%%%%%%%%%%%%%%%%%%%%%%
\definecolor{listinggray}{gray}{0.9}
\definecolor{lbcolor}{rgb}{0.9,0.9,0.9}
\definecolor{Darkgreen}{rgb}{0.13,0.545,0.13}

\usepackage[numbered,framed]{matlab-prettifier}

\usepackage{filecontents}

\let\ph\mlplaceholder % shorter macro
\lstMakeShortInline"

\lstset{
  style              = Matlab-editor,
  basicstyle         = \mlttfamily,
  escapechar         = ",
  mlshowsectionrules = true,
}


% Command for placeholder
\newcommand\placeholder[1]%
{%
    \bgroup
        \normalfont\upshape\color{RubineRed}%
        < {\itshape #1\/} >%
    \egroup
}


% Command for error
\newcommand\error[1]%
{%
    \bgroup
        \color{red}%
        #1\/
    \egroup
}



\lstset
{%
    escapechar=`,
}
%%%%%%%%%%%%%%%%%%%%%%%%%%%%%%%%%%%%%%%%%%%%%%%%%%%%%%%%%%%%%%%%%%%%%%%%%%%%%%%%%%%%%%%%
\begin{document}
\hrule

\begin{figure}[t]
	\begin{subfigure}[b]{0.40\linewidth} 					
    \includegraphics[height=0.8cm]{logo.jpg}     
  \end{subfigure}
    \hfill
	\begin{subfigure}[B]{0.40\linewidth} 						
    \includegraphics[height=0.8cm]{Pictures/bioengineering_logo_right.eps}
  \end{subfigure}
    \hfill
\end{figure}

\vspace{1cm}


\begin{center}
{\large \textbf{Mini-project 2019: Image processing}} % Title of your document
\\
\textsc{Marking scheme}
\end{center}
We are splitting the students in groups, each TA will be responsible for the marking of their group. Grouping can be done on the day depending on where people are sitting.
The students will be assessed during the session. They will have to show to their TA how their code works. The TAs will then assign a mark based on this mark scheme.


For both the rotation and the edge detection, if their code runs and outputs an image, check for any errors listed below and subtract marks as needed. If you spot any other mistake, subtract marks at your own discretion.
If their code does not run, then have a look to see if they have some of the basic structure needed (also listed below). Add up the marks for every thing they tried to implement.

\section{Rotation}

\begin{table}[H]
    \begin{tabular}{l|c}
    \textbf{Mistake}                                                                        & \textbf{Penalty} \\ \hline
    Image is flipped                                                               & -10     \\ \hline
    Output dimensions are wrong                                                    & -10     \\ \hline
    Rotation is wrong                                                              & -10     \\ \hline
    Image not centred                                                              & -10     \\ \hline
    Hard coded properties                                                          & -30     \\ \hline
    Unclear explanation of the code                                                & -20     \\ \hline
  
    \end{tabular}
\end{table}

If the code does not run:
\begin{table}[H]
    \begin{tabular}{l|c}
    \textbf{What they did}                                                                        & \textbf{Marks} \\ \hline
    Define the image transform                                                              & +10     \\ \hline
    Have some sort of loop to apply the transform to each pixel                             & +20     \\ \hline
    Attempt at assigning new pixel position after applying transform                              & +10     \\ \hline
    \end{tabular}
\end{table}
If it is a simple syntax error, you can subtract 5 marks and refer to first table.

\section{Edge detection}
\begin{table}[H]
    \begin{tabular}{l|c}
    \textbf{Mistake}                                                                        & \textbf{Penalty} \\ \hline
    Equation implemented incorrectly                                                               & -10     \\ \hline
    Incorrect handling of edge pixels                                                               & -10     \\ \hline
    Hard coded properties                                                          & -10     \\ \hline
  
    \end{tabular}
\end{table}

If the code does not run:
\begin{table}[H]
    \begin{tabular}{l|c}
    \textbf{What they did}                                                                        & \textbf{Marks} \\ \hline
    Have some sort of loop through each pixel                                                     & +10     \\ \hline
    \end{tabular}
\end{table}
If it is a simple syntax error, you can subtract 5 marks and refer to first table.

\section{Vectorisation}
They should get an extra 10 marks if they have successfully completed the vectorisation problem. Note that their score can not exceed 100.

\end{document}